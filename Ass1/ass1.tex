% Options for packages loaded elsewhere
\PassOptionsToPackage{unicode}{hyperref}
\PassOptionsToPackage{hyphens}{url}
%
\documentclass[
]{article}
\usepackage{amsmath,amssymb}
\usepackage{iftex}
\ifPDFTeX
  \usepackage[T1]{fontenc}
  \usepackage[utf8]{inputenc}
  \usepackage{textcomp} % provide euro and other symbols
\else % if luatex or xetex
  \usepackage{unicode-math} % this also loads fontspec
  \defaultfontfeatures{Scale=MatchLowercase}
  \defaultfontfeatures[\rmfamily]{Ligatures=TeX,Scale=1}
\fi
\usepackage{lmodern}
\ifPDFTeX\else
  % xetex/luatex font selection
\fi
% Use upquote if available, for straight quotes in verbatim environments
\IfFileExists{upquote.sty}{\usepackage{upquote}}{}
\IfFileExists{microtype.sty}{% use microtype if available
  \usepackage[]{microtype}
  \UseMicrotypeSet[protrusion]{basicmath} % disable protrusion for tt fonts
}{}
\makeatletter
\@ifundefined{KOMAClassName}{% if non-KOMA class
  \IfFileExists{parskip.sty}{%
    \usepackage{parskip}
  }{% else
    \setlength{\parindent}{0pt}
    \setlength{\parskip}{6pt plus 2pt minus 1pt}}
}{% if KOMA class
  \KOMAoptions{parskip=half}}
\makeatother
\usepackage{xcolor}
\usepackage[margin=1in]{geometry}
\usepackage{color}
\usepackage{fancyvrb}
\newcommand{\VerbBar}{|}
\newcommand{\VERB}{\Verb[commandchars=\\\{\}]}
\DefineVerbatimEnvironment{Highlighting}{Verbatim}{commandchars=\\\{\}}
% Add ',fontsize=\small' for more characters per line
\usepackage{framed}
\definecolor{shadecolor}{RGB}{248,248,248}
\newenvironment{Shaded}{\begin{snugshade}}{\end{snugshade}}
\newcommand{\AlertTok}[1]{\textcolor[rgb]{0.94,0.16,0.16}{#1}}
\newcommand{\AnnotationTok}[1]{\textcolor[rgb]{0.56,0.35,0.01}{\textbf{\textit{#1}}}}
\newcommand{\AttributeTok}[1]{\textcolor[rgb]{0.13,0.29,0.53}{#1}}
\newcommand{\BaseNTok}[1]{\textcolor[rgb]{0.00,0.00,0.81}{#1}}
\newcommand{\BuiltInTok}[1]{#1}
\newcommand{\CharTok}[1]{\textcolor[rgb]{0.31,0.60,0.02}{#1}}
\newcommand{\CommentTok}[1]{\textcolor[rgb]{0.56,0.35,0.01}{\textit{#1}}}
\newcommand{\CommentVarTok}[1]{\textcolor[rgb]{0.56,0.35,0.01}{\textbf{\textit{#1}}}}
\newcommand{\ConstantTok}[1]{\textcolor[rgb]{0.56,0.35,0.01}{#1}}
\newcommand{\ControlFlowTok}[1]{\textcolor[rgb]{0.13,0.29,0.53}{\textbf{#1}}}
\newcommand{\DataTypeTok}[1]{\textcolor[rgb]{0.13,0.29,0.53}{#1}}
\newcommand{\DecValTok}[1]{\textcolor[rgb]{0.00,0.00,0.81}{#1}}
\newcommand{\DocumentationTok}[1]{\textcolor[rgb]{0.56,0.35,0.01}{\textbf{\textit{#1}}}}
\newcommand{\ErrorTok}[1]{\textcolor[rgb]{0.64,0.00,0.00}{\textbf{#1}}}
\newcommand{\ExtensionTok}[1]{#1}
\newcommand{\FloatTok}[1]{\textcolor[rgb]{0.00,0.00,0.81}{#1}}
\newcommand{\FunctionTok}[1]{\textcolor[rgb]{0.13,0.29,0.53}{\textbf{#1}}}
\newcommand{\ImportTok}[1]{#1}
\newcommand{\InformationTok}[1]{\textcolor[rgb]{0.56,0.35,0.01}{\textbf{\textit{#1}}}}
\newcommand{\KeywordTok}[1]{\textcolor[rgb]{0.13,0.29,0.53}{\textbf{#1}}}
\newcommand{\NormalTok}[1]{#1}
\newcommand{\OperatorTok}[1]{\textcolor[rgb]{0.81,0.36,0.00}{\textbf{#1}}}
\newcommand{\OtherTok}[1]{\textcolor[rgb]{0.56,0.35,0.01}{#1}}
\newcommand{\PreprocessorTok}[1]{\textcolor[rgb]{0.56,0.35,0.01}{\textit{#1}}}
\newcommand{\RegionMarkerTok}[1]{#1}
\newcommand{\SpecialCharTok}[1]{\textcolor[rgb]{0.81,0.36,0.00}{\textbf{#1}}}
\newcommand{\SpecialStringTok}[1]{\textcolor[rgb]{0.31,0.60,0.02}{#1}}
\newcommand{\StringTok}[1]{\textcolor[rgb]{0.31,0.60,0.02}{#1}}
\newcommand{\VariableTok}[1]{\textcolor[rgb]{0.00,0.00,0.00}{#1}}
\newcommand{\VerbatimStringTok}[1]{\textcolor[rgb]{0.31,0.60,0.02}{#1}}
\newcommand{\WarningTok}[1]{\textcolor[rgb]{0.56,0.35,0.01}{\textbf{\textit{#1}}}}
\usepackage{graphicx}
\makeatletter
\def\maxwidth{\ifdim\Gin@nat@width>\linewidth\linewidth\else\Gin@nat@width\fi}
\def\maxheight{\ifdim\Gin@nat@height>\textheight\textheight\else\Gin@nat@height\fi}
\makeatother
% Scale images if necessary, so that they will not overflow the page
% margins by default, and it is still possible to overwrite the defaults
% using explicit options in \includegraphics[width, height, ...]{}
\setkeys{Gin}{width=\maxwidth,height=\maxheight,keepaspectratio}
% Set default figure placement to htbp
\makeatletter
\def\fps@figure{htbp}
\makeatother
\setlength{\emergencystretch}{3em} % prevent overfull lines
\providecommand{\tightlist}{%
  \setlength{\itemsep}{0pt}\setlength{\parskip}{0pt}}
\setcounter{secnumdepth}{-\maxdimen} % remove section numbering
\ifLuaTeX
  \usepackage{selnolig}  % disable illegal ligatures
\fi
\IfFileExists{bookmark.sty}{\usepackage{bookmark}}{\usepackage{hyperref}}
\IfFileExists{xurl.sty}{\usepackage{xurl}}{} % add URL line breaks if available
\urlstyle{same}
\hypersetup{
  pdftitle={Assignment 1},
  pdfauthor={Martynas Vaznonois, Andrei Puchkov, Carlo Peron (Group 1)},
  hidelinks,
  pdfcreator={LaTeX via pandoc}}

\title{Assignment 1}
\author{Martynas Vaznonois, Andrei Puchkov, Carlo Peron (Group 1)}
\date{23 February 2024}

\begin{document}
\maketitle

\hypertarget{exercise-1}{%
\subsection{Exercise 1}\label{exercise-1}}

\textbf{a)} The histogram and the qq plot look fairly indicative of
normality but the Shapiro-Wilk test provides evidence for the contrary.

\begin{Shaded}
\begin{Highlighting}[]
\FunctionTok{par}\NormalTok{(}\AttributeTok{mfrow=}\FunctionTok{c}\NormalTok{(}\DecValTok{1}\NormalTok{, }\DecValTok{2}\NormalTok{)); }\FunctionTok{qqnorm}\NormalTok{(data}\SpecialCharTok{$}\NormalTok{video)}
\FunctionTok{hist}\NormalTok{(data}\SpecialCharTok{$}\NormalTok{video, }\AttributeTok{freq=}\NormalTok{T, }\AttributeTok{main=}\StringTok{"Histogram of video"}\NormalTok{, }\AttributeTok{xlab=}\StringTok{"Video"}\NormalTok{)}
\end{Highlighting}
\end{Shaded}

\includegraphics{ass1_files/figure-latex/unnamed-chunk-1-1.pdf}

\begin{Shaded}
\begin{Highlighting}[]
\FunctionTok{shapiro.test}\NormalTok{(data}\SpecialCharTok{$}\NormalTok{video)}
\end{Highlighting}
\end{Shaded}

\begin{verbatim}
## 
##  Shapiro-Wilk normality test
## 
## data:  data$video
## W = 1, p-value = 0.03
\end{verbatim}

Assuming normality, especially with a sample size \(n>30\), we use the
z-score instead of the \(t_{n-1}\)-score to compute the 97\% confidence
interval. Rearranging the formula, we find the number of participants
needed to make the CI at most 3. Finally, we compute the bootstrap CI by
simulating sampling from the same distribution and calculating the mean
test statistics.

\begin{Shaded}
\begin{Highlighting}[]
\CommentTok{\# CI}
\NormalTok{alpha }\OtherTok{=} \FloatTok{0.03}
\NormalTok{ci }\OtherTok{\textless{}{-}} \FunctionTok{qnorm}\NormalTok{(}\DecValTok{1} \SpecialCharTok{{-}}\NormalTok{ alpha}\SpecialCharTok{/}\DecValTok{2}\NormalTok{) }\SpecialCharTok{*} \FunctionTok{sd}\NormalTok{(data}\SpecialCharTok{$}\NormalTok{video) }\SpecialCharTok{/} \FunctionTok{sqrt}\NormalTok{(}\FunctionTok{length}\NormalTok{(data}\SpecialCharTok{$}\NormalTok{video))}
\FunctionTok{c}\NormalTok{(}\FunctionTok{mean}\NormalTok{(data}\SpecialCharTok{$}\NormalTok{video) }\SpecialCharTok{{-}}\NormalTok{ ci, }\FunctionTok{mean}\NormalTok{(data}\SpecialCharTok{$}\NormalTok{video) }\SpecialCharTok{+}\NormalTok{ ci)}
\end{Highlighting}
\end{Shaded}

\begin{verbatim}
## [1] 50.3 53.4
\end{verbatim}

\begin{Shaded}
\begin{Highlighting}[]
\CommentTok{\# Participants}
\NormalTok{n }\OtherTok{\textless{}{-}}\NormalTok{ (}\FunctionTok{qnorm}\NormalTok{(}\DecValTok{1} \SpecialCharTok{{-}}\NormalTok{ alpha}\SpecialCharTok{/}\DecValTok{2}\NormalTok{) }\SpecialCharTok{*} \FunctionTok{sd}\NormalTok{(data}\SpecialCharTok{$}\NormalTok{video) }\SpecialCharTok{/}\NormalTok{ (}\DecValTok{3}\SpecialCharTok{/}\DecValTok{2}\NormalTok{))}\SpecialCharTok{\^{}}\DecValTok{2}\NormalTok{; }\FunctionTok{ceiling}\NormalTok{(n)}
\end{Highlighting}
\end{Shaded}

\begin{verbatim}
## [1] 206
\end{verbatim}

\begin{Shaded}
\begin{Highlighting}[]
\CommentTok{\# Bootstrap CI}
\NormalTok{B }\OtherTok{=} \DecValTok{100000}
\NormalTok{Tstar }\OtherTok{=} \FunctionTok{numeric}\NormalTok{(B)}
\ControlFlowTok{for}\NormalTok{(i }\ControlFlowTok{in} \DecValTok{1}\SpecialCharTok{:}\NormalTok{B)}
\NormalTok{  Tstar[i] }\OtherTok{=} \FunctionTok{mean}\NormalTok{(}\FunctionTok{sample}\NormalTok{(data}\SpecialCharTok{$}\NormalTok{video, }\AttributeTok{replace=}\NormalTok{T))}
\FunctionTok{c}\NormalTok{(}\DecValTok{2}\SpecialCharTok{*}\FunctionTok{mean}\NormalTok{(Tstar) }\SpecialCharTok{{-}} \FunctionTok{quantile}\NormalTok{(Tstar, }\DecValTok{1} \SpecialCharTok{{-}}\NormalTok{ alpha}\SpecialCharTok{/}\DecValTok{2}\NormalTok{), }\DecValTok{2}\SpecialCharTok{*}\FunctionTok{mean}\NormalTok{(Tstar) }\SpecialCharTok{{-}} \FunctionTok{quantile}\NormalTok{(Tstar, alpha}\SpecialCharTok{/}\DecValTok{2}\NormalTok{))}
\end{Highlighting}
\end{Shaded}

\begin{verbatim}
## 98.5%  1.5% 
##  50.4  53.4
\end{verbatim}

\textbf{b)} The first test below shows that the \(H_0\) is rejected
(\(p<0.001\)). The confidence interval for the true mean is
\((50.69; \infty)\). This shows that the true mean can be find within
that interval with 95\% certainty. Because the mean of 50 under \(H_0\)
is not in that interval, the probability of \(H_0\) being true is less
than 5\%, leading ti its rejection. Under the second test, the \(\mu_0\)
is within that same interval, which is why the p-value is above 0.05 and
why \(H_0\) cannot be rejected.

\begin{Shaded}
\begin{Highlighting}[]
\FunctionTok{t.test}\NormalTok{(data}\SpecialCharTok{$}\NormalTok{video, }\AttributeTok{mu=}\DecValTok{50}\NormalTok{, }\AttributeTok{alternative=}\StringTok{"greater"}\NormalTok{)}
\end{Highlighting}
\end{Shaded}

\begin{verbatim}
## 
##  One Sample t-test
## 
## data:  data$video
## t = 3, df = 199, p-value = 0.004
## alternative hypothesis: true mean is greater than 50
## 95 percent confidence interval:
##  50.7  Inf
## sample estimates:
## mean of x 
##      51.9
\end{verbatim}

\begin{Shaded}
\begin{Highlighting}[]
\FunctionTok{t.test}\NormalTok{(data}\SpecialCharTok{$}\NormalTok{video, }\AttributeTok{mu=}\DecValTok{51}\NormalTok{, }\AttributeTok{alternative=}\StringTok{"greater"}\NormalTok{)}
\end{Highlighting}
\end{Shaded}

\begin{verbatim}
## 
##  One Sample t-test
## 
## data:  data$video
## t = 1, df = 199, p-value = 0.1
## alternative hypothesis: true mean is greater than 51
## 95 percent confidence interval:
##  50.7  Inf
## sample estimates:
## mean of x 
##      51.9
\end{verbatim}

\textbf{c)} The first test below is the sign (binomial) test. It assumes
\(H_0: m=50\) to be the median and if the \(H_0\) is correct, then there
should be approximately the same number of observations to the right and
left of it. The Wilcoxon test also considers ranks to make stronger
predictions. Comparing to \textbf{b)}, the t-tests require the data to
be normally distributed which is unclear in our case. The two latter
tests do not make such an assumption which may make them more
applicable. On the other hand, they throw away more information, and in
the case of the sign test, so much information has been lost that
\(H_0\) cannot be rejected, like it was with the t-test. The Wilcoxon
test only assumes symmetrical distribution, which seems likely. And it,
like the t-test, provides enough evidence to reject \(H_0\)

\begin{Shaded}
\begin{Highlighting}[]
\FunctionTok{binom.test}\NormalTok{(}\FunctionTok{sum}\NormalTok{(data}\SpecialCharTok{$}\NormalTok{video}\SpecialCharTok{\textgreater{}}\DecValTok{50}\NormalTok{), }\FunctionTok{length}\NormalTok{(data}\SpecialCharTok{$}\NormalTok{video), }\AttributeTok{alternative=}\StringTok{\textquotesingle{}g\textquotesingle{}}\NormalTok{)}
\end{Highlighting}
\end{Shaded}

\begin{verbatim}
## 
##  Exact binomial test
## 
## data:  sum(data$video > 50) and length(data$video)
## number of successes = 108, number of trials = 200, p-value = 0.1
## alternative hypothesis: true probability of success is greater than 0.5
## 95 percent confidence interval:
##  0.479 1.000
## sample estimates:
## probability of success 
##                   0.54
\end{verbatim}

\begin{Shaded}
\begin{Highlighting}[]
\FunctionTok{wilcox.test}\NormalTok{(data}\SpecialCharTok{$}\NormalTok{video, }\AttributeTok{mu=}\DecValTok{50}\NormalTok{, }\AttributeTok{alternative=}\StringTok{\textquotesingle{}g\textquotesingle{}}\NormalTok{)}
\end{Highlighting}
\end{Shaded}

\begin{verbatim}
## 
##  Wilcoxon signed rank test with continuity correction
## 
## data:  data$video
## V = 9836, p-value = 0.005
## alternative hypothesis: true location is greater than 50
\end{verbatim}

Testing whether less than 25\% of the data falls below 42 comes down to
calculating the fraction of the data below 42 and checking if it is less
than 0.25.

\begin{Shaded}
\begin{Highlighting}[]
\FunctionTok{sum}\NormalTok{(data}\SpecialCharTok{$}\NormalTok{video}\SpecialCharTok{\textless{}}\DecValTok{42}\NormalTok{)}\SpecialCharTok{/}\FunctionTok{length}\NormalTok{(data}\SpecialCharTok{$}\NormalTok{video) }\SpecialCharTok{\textless{}=} \FloatTok{0.25}
\end{Highlighting}
\end{Shaded}

\begin{verbatim}
## [1] TRUE
\end{verbatim}

\textbf{d)}

\begin{Shaded}
\begin{Highlighting}[]
\CommentTok{\# Bootstrap test}
\NormalTok{B }\OtherTok{=} \DecValTok{10000}
\NormalTok{t }\OtherTok{=} \FunctionTok{min}\NormalTok{(data}\SpecialCharTok{$}\NormalTok{video)}
\NormalTok{Tstar }\OtherTok{=} \FunctionTok{numeric}\NormalTok{(B)}
\NormalTok{means }\OtherTok{=} \ConstantTok{NULL}
\ControlFlowTok{for}\NormalTok{(m }\ControlFlowTok{in} \DecValTok{0}\SpecialCharTok{:}\DecValTok{100}\NormalTok{)\{}
  \ControlFlowTok{for}\NormalTok{(i }\ControlFlowTok{in} \DecValTok{1}\SpecialCharTok{:}\NormalTok{B)}
\NormalTok{    Tstar[i] }\OtherTok{=} \FunctionTok{min}\NormalTok{(}\FunctionTok{rnorm}\NormalTok{(}\FunctionTok{length}\NormalTok{(data}\SpecialCharTok{$}\NormalTok{video), }\AttributeTok{mean=}\NormalTok{m, }\AttributeTok{sd=}\DecValTok{10}\NormalTok{))}
  \ControlFlowTok{if}\NormalTok{(}\DecValTok{2}\SpecialCharTok{*}\FunctionTok{min}\NormalTok{(}\FunctionTok{sum}\NormalTok{(Tstar}\SpecialCharTok{\textless{}}\NormalTok{t)}\SpecialCharTok{/}\NormalTok{B, }\FunctionTok{sum}\NormalTok{(Tstar}\SpecialCharTok{\textgreater{}}\NormalTok{t)}\SpecialCharTok{/}\NormalTok{B) }\SpecialCharTok{\textgreater{}} \FloatTok{0.05}\NormalTok{)}
\NormalTok{    means }\OtherTok{=} \FunctionTok{c}\NormalTok{(means, m)}
\NormalTok{\}}
\FunctionTok{range}\NormalTok{(means)}
\end{Highlighting}
\end{Shaded}

\begin{verbatim}
## [1] 47 62
\end{verbatim}

Kolmogorov-Smirnov test, to roughly estimate the means which fit

\begin{Shaded}
\begin{Highlighting}[]
\NormalTok{means }\OtherTok{=} \ConstantTok{NULL}
\ControlFlowTok{for}\NormalTok{(m }\ControlFlowTok{in} \DecValTok{0}\SpecialCharTok{:}\DecValTok{100}\NormalTok{)\{}
  \ControlFlowTok{if}\NormalTok{(}\FunctionTok{ks.test}\NormalTok{(data}\SpecialCharTok{$}\NormalTok{video, pnorm, m, }\DecValTok{10}\NormalTok{)[[}\DecValTok{2}\NormalTok{]] }\SpecialCharTok{\textgreater{}} \FloatTok{0.05}\NormalTok{)}
\NormalTok{    means }\OtherTok{=} \FunctionTok{c}\NormalTok{(means, m)}
\NormalTok{\}}
\FunctionTok{range}\NormalTok{(means)}
\end{Highlighting}
\end{Shaded}

\begin{verbatim}
## [1] 52 53
\end{verbatim}

More precise KS test

\begin{Shaded}
\begin{Highlighting}[]
\NormalTok{means }\OtherTok{=} \ConstantTok{NULL}
\ControlFlowTok{for}\NormalTok{(m }\ControlFlowTok{in} \FunctionTok{seq}\NormalTok{(}\DecValTok{51}\NormalTok{, }\DecValTok{54}\NormalTok{, }\AttributeTok{by=}\FloatTok{0.001}\NormalTok{))\{}
  \ControlFlowTok{if}\NormalTok{(}\FunctionTok{ks.test}\NormalTok{(data}\SpecialCharTok{$}\NormalTok{video, pnorm, m, }\DecValTok{10}\NormalTok{)[[}\DecValTok{2}\NormalTok{]] }\SpecialCharTok{\textgreater{}} \FloatTok{0.05}\NormalTok{)}
\NormalTok{    means }\OtherTok{=} \FunctionTok{c}\NormalTok{(means, m)}
\NormalTok{\}}
\FunctionTok{range}\NormalTok{(means)}
\end{Highlighting}
\end{Shaded}

\begin{verbatim}
## [1] 51.3 53.4
\end{verbatim}

\textbf{e)}

Here we create separate dataframes for male scores and female ones

\begin{Shaded}
\begin{Highlighting}[]
\NormalTok{fvideo }\OtherTok{=}\NormalTok{ data}\SpecialCharTok{$}\NormalTok{video[data}\SpecialCharTok{$}\NormalTok{female }\SpecialCharTok{==} \DecValTok{1}\NormalTok{]}
\NormalTok{mvideo }\OtherTok{=}\NormalTok{ data}\SpecialCharTok{$}\NormalTok{video[data}\SpecialCharTok{$}\NormalTok{female }\SpecialCharTok{==} \DecValTok{0}\NormalTok{]}
\end{Highlighting}
\end{Shaded}

Then we run 3 tests on the data to check if male scores are indeed
higher than female.

\begin{Shaded}
\begin{Highlighting}[]
\FunctionTok{t.test}\NormalTok{(mvideo, fvideo, }\AttributeTok{alternative=}\StringTok{\textquotesingle{}g\textquotesingle{}}\NormalTok{)}
\end{Highlighting}
\end{Shaded}

\begin{verbatim}
## 
##  Welch Two Sample t-test
## 
## data:  mvideo and fvideo
## t = 2, df = 177, p-value = 0.04
## alternative hypothesis: true difference in means is greater than 0
## 95 percent confidence interval:
##  0.186   Inf
## sample estimates:
## mean of x mean of y 
##      53.2      50.7
\end{verbatim}

\begin{Shaded}
\begin{Highlighting}[]
\FunctionTok{wilcox.test}\NormalTok{(mvideo, fvideo, }\AttributeTok{alternative=}\StringTok{\textquotesingle{}g\textquotesingle{}}\NormalTok{)}
\end{Highlighting}
\end{Shaded}

\begin{verbatim}
## 
##  Wilcoxon rank sum test with continuity correction
## 
## data:  mvideo and fvideo
## W = 5748, p-value = 0.03
## alternative hypothesis: true location shift is greater than 0
\end{verbatim}

\begin{Shaded}
\begin{Highlighting}[]
\FunctionTok{ks.test}\NormalTok{(mvideo, fvideo, }\AttributeTok{alternative=}\StringTok{\textquotesingle{}l\textquotesingle{}}\NormalTok{)}
\end{Highlighting}
\end{Shaded}

\begin{verbatim}
## 
##  Exact two-sample Kolmogorov-Smirnov test
## 
## data:  mvideo and fvideo
## D^- = 0.2, p-value = 0.04
## alternative hypothesis: the CDF of x lies below that of y
\end{verbatim}

Which means the expert was right.

To answer about tests' applicability, we need to check the distribution
of data. For t-test it needs to follow the \textcolor{red}{normal
distribution}.

\begin{Shaded}
\begin{Highlighting}[]
\FunctionTok{par}\NormalTok{(}\AttributeTok{mfrow=}\FunctionTok{c}\NormalTok{(}\DecValTok{2}\NormalTok{, }\DecValTok{2}\NormalTok{))}
\FunctionTok{qqnorm}\NormalTok{(fvideo); }\FunctionTok{qqnorm}\NormalTok{(mvideo); }\FunctionTok{hist}\NormalTok{(fvideo); }\FunctionTok{hist}\NormalTok{(mvideo)}
\end{Highlighting}
\end{Shaded}

\includegraphics{ass1_files/figure-latex/unnamed-chunk-11-1.pdf}

\begin{Shaded}
\begin{Highlighting}[]
\FunctionTok{shapiro.test}\NormalTok{(fvideo); }\FunctionTok{shapiro.test}\NormalTok{(mvideo)}
\end{Highlighting}
\end{Shaded}

\begin{verbatim}
## 
##  Shapiro-Wilk normality test
## 
## data:  fvideo
## W = 1, p-value = 0.3
\end{verbatim}

\begin{verbatim}
## 
##  Shapiro-Wilk normality test
## 
## data:  mvideo
## W = 1, p-value = 0.06
\end{verbatim}

The test shows that the data is indeed normally distributed, although
female video results are on the borderline.

\textbf{f)} Here we apply Pearson correlation test along with Spearman
test, which shows monotonic relationship between variables. Pearson
shows just linear dependency in data.

\begin{Shaded}
\begin{Highlighting}[]
\FunctionTok{cor.test}\NormalTok{(data}\SpecialCharTok{$}\NormalTok{video, data}\SpecialCharTok{$}\NormalTok{puzzle)}
\end{Highlighting}
\end{Shaded}

\begin{verbatim}
## 
##  Pearson's product-moment correlation
## 
## data:  data$video and data$puzzle
## t = 7, df = 198, p-value = 4e-12
## alternative hypothesis: true correlation is not equal to 0
## 95 percent confidence interval:
##  0.349 0.567
## sample estimates:
##   cor 
## 0.465
\end{verbatim}

\begin{Shaded}
\begin{Highlighting}[]
\FunctionTok{cor.test}\NormalTok{(data}\SpecialCharTok{$}\NormalTok{video, data}\SpecialCharTok{$}\NormalTok{puzzle, }\AttributeTok{method=}\StringTok{\textquotesingle{}spearman\textquotesingle{}}\NormalTok{)}
\end{Highlighting}
\end{Shaded}

\begin{verbatim}
## 
##  Spearman's rank correlation rho
## 
## data:  data$video and data$puzzle
## S = 7e+05, p-value = 6e-13
## alternative hypothesis: true rho is not equal to 0
## sample estimates:
##   rho 
## 0.481
\end{verbatim}

We see that data is weakly correlated.

Then we check the data for symmetricity and normality

\begin{Shaded}
\begin{Highlighting}[]
\FunctionTok{par}\NormalTok{(}\AttributeTok{mfrow=}\FunctionTok{c}\NormalTok{(}\DecValTok{2}\NormalTok{, }\DecValTok{2}\NormalTok{))}
\FunctionTok{qqnorm}\NormalTok{(data}\SpecialCharTok{$}\NormalTok{video); }\FunctionTok{qqnorm}\NormalTok{(data}\SpecialCharTok{$}\NormalTok{puzzle);}
\FunctionTok{hist}\NormalTok{(data}\SpecialCharTok{$}\NormalTok{video); }\FunctionTok{hist}\NormalTok{(data}\SpecialCharTok{$}\NormalTok{puzzle)}
\end{Highlighting}
\end{Shaded}

\includegraphics{ass1_files/figure-latex/unnamed-chunk-13-1.pdf}

\begin{Shaded}
\begin{Highlighting}[]
\FunctionTok{shapiro.test}\NormalTok{(data}\SpecialCharTok{$}\NormalTok{video); }\FunctionTok{shapiro.test}\NormalTok{(data}\SpecialCharTok{$}\NormalTok{puzzle)}
\end{Highlighting}
\end{Shaded}

\begin{verbatim}
## 
##  Shapiro-Wilk normality test
## 
## data:  data$video
## W = 1, p-value = 0.03
\end{verbatim}

\begin{verbatim}
## 
##  Shapiro-Wilk normality test
## 
## data:  data$puzzle
## W = 1, p-value = 2e-05
\end{verbatim}

\hypertarget{exercise-2}{%
\subsection{Exercise 2}\label{exercise-2}}

\begin{Shaded}
\begin{Highlighting}[]
\NormalTok{data }\OtherTok{\textless{}{-}} \FunctionTok{read.table}\NormalTok{(}\StringTok{"hemoglobin{-}1.txt"}\NormalTok{, }\AttributeTok{header=}\NormalTok{T)}
\NormalTok{data}\SpecialCharTok{$}\NormalTok{rate }\OtherTok{\textless{}{-}} \FunctionTok{as.factor}\NormalTok{(data}\SpecialCharTok{$}\NormalTok{rate)}
\NormalTok{data}\SpecialCharTok{$}\NormalTok{method }\OtherTok{\textless{}{-}} \FunctionTok{as.factor}\NormalTok{(data}\SpecialCharTok{$}\NormalTok{method)}
\end{Highlighting}
\end{Shaded}

\textbf{a)}

\textbf{b)}

\begin{Shaded}
\begin{Highlighting}[]
\FunctionTok{anova}\NormalTok{(}\FunctionTok{lm}\NormalTok{(hemoglobin}\SpecialCharTok{\textasciitilde{}}\NormalTok{rate}\SpecialCharTok{*}\NormalTok{method, }\AttributeTok{data=}\NormalTok{data))}
\end{Highlighting}
\end{Shaded}

\begin{verbatim}
## Analysis of Variance Table
## 
## Response: hemoglobin
##             Df Sum Sq Mean Sq F value  Pr(>F)    
## rate         3   90.6   30.19   19.47 2.4e-09 ***
## method       1    2.4    2.42    1.56    0.22    
## rate:method  3    4.9    1.62    1.05    0.38    
## Residuals   72  111.6    1.55                    
## ---
## Signif. codes:  0 '***' 0.001 '**' 0.01 '*' 0.05 '.' 0.1 ' ' 1
\end{verbatim}

Here we see that \textbf{method} does not have a significant impact on
hemoglobin levels, while \textbf{rate} does.
\textcolor{red}{comment on findings}

\textbf{c)}

\begin{Shaded}
\begin{Highlighting}[]
\FunctionTok{anova}\NormalTok{(}\FunctionTok{lm}\NormalTok{(hemoglobin}\SpecialCharTok{\textasciitilde{}}\NormalTok{rate}\SpecialCharTok{+}\NormalTok{method, }\AttributeTok{data=}\NormalTok{data))}
\end{Highlighting}
\end{Shaded}

\begin{verbatim}
## Analysis of Variance Table
## 
## Response: hemoglobin
##           Df Sum Sq Mean Sq F value Pr(>F)    
## rate       3   90.6   30.19   19.43  2e-09 ***
## method     1    2.4    2.42    1.55   0.22    
## Residuals 75  116.5    1.55                   
## ---
## Signif. codes:  0 '***' 0.001 '**' 0.01 '*' 0.05 '.' 0.1 ' ' 1
\end{verbatim}

\begin{Shaded}
\begin{Highlighting}[]
\FunctionTok{summary}\NormalTok{(}\FunctionTok{lm}\NormalTok{(hemoglobin}\SpecialCharTok{\textasciitilde{}}\NormalTok{rate}\SpecialCharTok{+}\NormalTok{method, }\AttributeTok{data=}\NormalTok{data))}
\end{Highlighting}
\end{Shaded}

\begin{verbatim}
## 
## Call:
## lm(formula = hemoglobin ~ rate + method, data = data)
## 
## Residuals:
##    Min     1Q Median     3Q    Max 
## -3.454 -0.888  0.005  0.841  2.339 
## 
## Coefficients:
##             Estimate Std. Error t value Pr(>|t|)    
## (Intercept)    6.801      0.312   21.83  < 2e-16 ***
## rate2          2.760      0.394    7.00  9.2e-10 ***
## rate3          2.405      0.394    6.10  4.2e-08 ***
## rate4          1.880      0.394    4.77  8.9e-06 ***
## methodB        0.348      0.279    1.25     0.22    
## ---
## Signif. codes:  0 '***' 0.001 '**' 0.01 '*' 0.05 '.' 0.1 ' ' 1
## 
## Residual standard error: 1.25 on 75 degrees of freedom
## Multiple R-squared:  0.444,  Adjusted R-squared:  0.414 
## F-statistic:   15 on 4 and 75 DF,  p-value: 4.92e-09
\end{verbatim}

As we see from the previous point, \textbf{rate} has greater impact
compared to method.

\begin{Shaded}
\begin{Highlighting}[]
\FunctionTok{mean}\NormalTok{(data}\SpecialCharTok{$}\NormalTok{hemoglobin[data}\SpecialCharTok{$}\NormalTok{rate}\SpecialCharTok{==}\DecValTok{3} \SpecialCharTok{\&}\NormalTok{ data}\SpecialCharTok{$}\NormalTok{method}\SpecialCharTok{==}\StringTok{\textquotesingle{}A\textquotesingle{}}\NormalTok{])}
\end{Highlighting}
\end{Shaded}

\begin{verbatim}
## [1] 9.03
\end{verbatim}

\begin{Shaded}
\begin{Highlighting}[]
\FunctionTok{which.max}\NormalTok{(}\FunctionTok{aggregate}\NormalTok{(hemoglobin }\SpecialCharTok{\textasciitilde{}}\NormalTok{ rate, }\AttributeTok{data=}\NormalTok{data, mean)}\SpecialCharTok{$}\NormalTok{hemoglobin)}
\end{Highlighting}
\end{Shaded}

\begin{verbatim}
## [1] 2
\end{verbatim}

Rate \textbf{2} leads to highest hemoglobin levels, while mean level for
rate 3 with method A is 9.03

\textbf{d)}

\begin{Shaded}
\begin{Highlighting}[]
\FunctionTok{anova}\NormalTok{(}\FunctionTok{lm}\NormalTok{(hemoglobin }\SpecialCharTok{\textasciitilde{}}\NormalTok{ rate, }\AttributeTok{data=}\NormalTok{data))}
\end{Highlighting}
\end{Shaded}

\begin{verbatim}
## Analysis of Variance Table
## 
## Response: hemoglobin
##           Df Sum Sq Mean Sq F value  Pr(>F)    
## rate       3   90.6   30.19    19.3 2.1e-09 ***
## Residuals 76  118.9    1.56                    
## ---
## Signif. codes:  0 '***' 0.001 '**' 0.01 '*' 0.05 '.' 0.1 ' ' 1
\end{verbatim}

\begin{Shaded}
\begin{Highlighting}[]
\FunctionTok{mean}\NormalTok{(data}\SpecialCharTok{$}\NormalTok{hemoglobin[data}\SpecialCharTok{$}\NormalTok{rate}\SpecialCharTok{==}\DecValTok{1}\NormalTok{])}
\end{Highlighting}
\end{Shaded}

\begin{verbatim}
## [1] 6.97
\end{verbatim}

\begin{Shaded}
\begin{Highlighting}[]
\FunctionTok{mean}\NormalTok{(data}\SpecialCharTok{$}\NormalTok{hemoglobin[data}\SpecialCharTok{$}\NormalTok{rate}\SpecialCharTok{==}\DecValTok{2}\NormalTok{])}
\end{Highlighting}
\end{Shaded}

\begin{verbatim}
## [1] 9.73
\end{verbatim}

\begin{Shaded}
\begin{Highlighting}[]
\FunctionTok{mean}\NormalTok{(data}\SpecialCharTok{$}\NormalTok{hemoglobin[data}\SpecialCharTok{$}\NormalTok{rate}\SpecialCharTok{==}\DecValTok{3}\NormalTok{])}
\end{Highlighting}
\end{Shaded}

\begin{verbatim}
## [1] 9.38
\end{verbatim}

\begin{Shaded}
\begin{Highlighting}[]
\FunctionTok{mean}\NormalTok{(data}\SpecialCharTok{$}\NormalTok{hemoglobin[data}\SpecialCharTok{$}\NormalTok{rate}\SpecialCharTok{==}\DecValTok{4}\NormalTok{])}
\end{Highlighting}
\end{Shaded}

\begin{verbatim}
## [1] 8.86
\end{verbatim}

\textcolor{red}{Is it right/wrong or useful/not useful to perform this test on this dataset?}

\textbf{e)}

\begin{Shaded}
\begin{Highlighting}[]
\FunctionTok{kruskal.test}\NormalTok{(hemoglobin }\SpecialCharTok{\textasciitilde{}}\NormalTok{ rate, }\AttributeTok{data=}\NormalTok{data)}
\end{Highlighting}
\end{Shaded}

\begin{verbatim}
## 
##  Kruskal-Wallis rank sum test
## 
## data:  hemoglobin by rate
## Kruskal-Wallis chi-squared = 34, df = 3, p-value = 2e-07
\end{verbatim}

\textcolor{red}{Explain possible differences between the Kruskal-Wallis and ANOVA tests.}
- Kruskal-Wallis is rank-based test while ANOVA is values (?) based.

\hypertarget{exercise-3}{%
\subsection{Exercise 3}\label{exercise-3}}

\begin{Shaded}
\begin{Highlighting}[]
\CommentTok{\# library(lme4)}
\NormalTok{data }\OtherTok{\textless{}{-}} \FunctionTok{read.table}\NormalTok{(}\StringTok{"cream{-}1.txt"}\NormalTok{, }\AttributeTok{header=}\NormalTok{T)}
\NormalTok{data}\SpecialCharTok{$}\NormalTok{batch }\OtherTok{\textless{}{-}} \FunctionTok{as.factor}\NormalTok{(data}\SpecialCharTok{$}\NormalTok{batch)}
\NormalTok{data}\SpecialCharTok{$}\NormalTok{position }\OtherTok{\textless{}{-}} \FunctionTok{as.factor}\NormalTok{(data}\SpecialCharTok{$}\NormalTok{position)}
\NormalTok{data}\SpecialCharTok{$}\NormalTok{starter }\OtherTok{\textless{}{-}} \FunctionTok{as.factor}\NormalTok{(data}\SpecialCharTok{$}\NormalTok{starter)}
\end{Highlighting}
\end{Shaded}

\textbf{a)}

\begin{Shaded}
\begin{Highlighting}[]
\NormalTok{aciditylm }\OtherTok{=} \FunctionTok{lm}\NormalTok{(acidity }\SpecialCharTok{\textasciitilde{}}\NormalTok{ batch }\SpecialCharTok{+}\NormalTok{ position }\SpecialCharTok{+}\NormalTok{ starter, data)}
\FunctionTok{anova}\NormalTok{(aciditylm)}
\end{Highlighting}
\end{Shaded}

\begin{verbatim}
## Analysis of Variance Table
## 
## Response: acidity
##           Df Sum Sq Mean Sq F value  Pr(>F)    
## batch      4   18.8    4.69    8.60  0.0016 ** 
## position   4    2.3    0.59    1.08  0.4112    
## starter    4   44.1   11.03   20.21 2.9e-05 ***
## Residuals 12    6.6    0.55                    
## ---
## Signif. codes:  0 '***' 0.001 '**' 0.01 '*' 0.05 '.' 0.1 ' ' 1
\end{verbatim}

\begin{Shaded}
\begin{Highlighting}[]
\FunctionTok{summary}\NormalTok{(aciditylm)}
\end{Highlighting}
\end{Shaded}

\begin{verbatim}
## 
## Call:
## lm(formula = acidity ~ batch + position + starter, data = data)
## 
## Residuals:
##     Min      1Q  Median      3Q     Max 
## -1.2836 -0.2336  0.0384  0.3584  1.0204 
## 
## Coefficients:
##             Estimate Std. Error t value Pr(>|t|)    
## (Intercept)    8.662      0.533   16.26  1.5e-09 ***
## batch2        -1.348      0.467   -2.88    0.014 *  
## batch3         0.276      0.467    0.59    0.566    
## batch4         1.368      0.467    2.93    0.013 *  
## batch5         0.200      0.467    0.43    0.676    
## position2     -0.618      0.467   -1.32    0.211    
## position3     -0.038      0.467   -0.08    0.937    
## position4     -0.764      0.467   -1.63    0.128    
## position5     -0.264      0.467   -0.56    0.583    
## starter2      -0.150      0.467   -0.32    0.754    
## starter3      -0.980      0.467   -2.10    0.058 .  
## starter4       2.810      0.467    6.01  6.1e-05 ***
## starter5      -0.484      0.467   -1.04    0.321    
## ---
## Signif. codes:  0 '***' 0.001 '**' 0.01 '*' 0.05 '.' 0.1 ' ' 1
## 
## Residual standard error: 0.739 on 12 degrees of freedom
## Multiple R-squared:  0.909,  Adjusted R-squared:  0.818 
## F-statistic: 9.96 on 12 and 12 DF,  p-value: 0.000178
\end{verbatim}

We cannot say that there is a statistically significant difference in
effects of starters 1 and 2 from this test.
\textcolor{red}{Motivate your answer} - p-value is high? Is that enough?

\textbf{b)}

\begin{Shaded}
\begin{Highlighting}[]
\NormalTok{aciditylm }\OtherTok{=} \FunctionTok{lm}\NormalTok{(acidity }\SpecialCharTok{\textasciitilde{}}\NormalTok{ batch }\SpecialCharTok{+}\NormalTok{ starter, data)}
\FunctionTok{anova}\NormalTok{(aciditylm)}
\end{Highlighting}
\end{Shaded}

\begin{verbatim}
## Analysis of Variance Table
## 
## Response: acidity
##           Df Sum Sq Mean Sq F value  Pr(>F)    
## batch      4   18.8    4.69    8.44 0.00073 ***
## starter    4   44.1   11.03   19.84 4.8e-06 ***
## Residuals 16    8.9    0.56                    
## ---
## Signif. codes:  0 '***' 0.001 '**' 0.01 '*' 0.05 '.' 0.1 ' ' 1
\end{verbatim}

\begin{Shaded}
\begin{Highlighting}[]
\FunctionTok{summary}\NormalTok{(aciditylm)}
\end{Highlighting}
\end{Shaded}

\begin{verbatim}
## 
## Call:
## lm(formula = acidity ~ batch + starter, data = data)
## 
## Residuals:
##     Min      1Q  Median      3Q     Max 
## -1.5648 -0.2548 -0.0548  0.3592  1.1352 
## 
## Coefficients:
##             Estimate Std. Error t value Pr(>|t|)    
## (Intercept)    8.325      0.447   18.60  2.9e-12 ***
## batch2        -1.348      0.472   -2.86    0.011 *  
## batch3         0.276      0.472    0.59    0.567    
## batch4         1.368      0.472    2.90    0.010 *  
## batch5         0.200      0.472    0.42    0.677    
## starter2      -0.150      0.472   -0.32    0.755    
## starter3      -0.980      0.472   -2.08    0.054 .  
## starter4       2.810      0.472    5.96  2.0e-05 ***
## starter5      -0.484      0.472   -1.03    0.320    
## ---
## Signif. codes:  0 '***' 0.001 '**' 0.01 '*' 0.05 '.' 0.1 ' ' 1
## 
## Residual standard error: 0.746 on 16 degrees of freedom
## Multiple R-squared:  0.876,  Adjusted R-squared:  0.814 
## F-statistic: 14.1 on 8 and 16 DF,  p-value: 6.47e-06
\end{verbatim}

Here we see that \textbf{starter 4} has the biggest effect on
\textbf{acidity}. \textcolor{red}{Motivvate your answer} The test shows
largest coefficient value for that starter as well as p-value is very
close to zero.

\textbf{c)}

\begin{Shaded}
\begin{Highlighting}[]
\FunctionTok{friedman.test}\NormalTok{(acidity }\SpecialCharTok{\textasciitilde{}}\NormalTok{ starter }\SpecialCharTok{|}\NormalTok{ batch, data)}
\end{Highlighting}
\end{Shaded}

\begin{verbatim}
## 
##  Friedman rank sum test
## 
## data:  acidity and starter and batch
## Friedman chi-squared = 13, df = 4, p-value = 0.01
\end{verbatim}

Friedman test is applicable here, as it does not assume normality in
data distribution, while 2-way ANOVA does.

\textbf{d)}

\begin{Shaded}
\begin{Highlighting}[]
\FunctionTok{library}\NormalTok{(lme4)}
\end{Highlighting}
\end{Shaded}

\begin{verbatim}
## Загрузка требуемого пакета: Matrix
\end{verbatim}

\begin{Shaded}
\begin{Highlighting}[]
\NormalTok{lm1 }\OtherTok{=} \FunctionTok{lmer}\NormalTok{(acidity }\SpecialCharTok{\textasciitilde{}}\NormalTok{ starter }\SpecialCharTok{+}\NormalTok{ (}\DecValTok{1}\SpecialCharTok{|}\NormalTok{batch) }\SpecialCharTok{+}\NormalTok{ (}\DecValTok{1}\SpecialCharTok{|}\NormalTok{position), data)}
\NormalTok{lm2 }\OtherTok{=} \FunctionTok{lmer}\NormalTok{(acidity }\SpecialCharTok{\textasciitilde{}}\NormalTok{ (}\DecValTok{1}\SpecialCharTok{|}\NormalTok{batch) }\SpecialCharTok{+}\NormalTok{ (}\DecValTok{1}\SpecialCharTok{|}\NormalTok{position), data)}
\end{Highlighting}
\end{Shaded}

\begin{verbatim}
## boundary (singular) fit: see help('isSingular')
\end{verbatim}

\begin{Shaded}
\begin{Highlighting}[]
\FunctionTok{anova}\NormalTok{(lm1, lm2)}
\end{Highlighting}
\end{Shaded}

\begin{verbatim}
## refitting model(s) with ML (instead of REML)
\end{verbatim}

\begin{verbatim}
## Data: data
## Models:
## lm2: acidity ~ (1 | batch) + (1 | position)
## lm1: acidity ~ starter + (1 | batch) + (1 | position)
##     npar   AIC   BIC logLik deviance Chisq Df Pr(>Chisq)    
## lm2    4 105.1 109.9  -48.5     97.1                        
## lm1    8  77.2  86.9  -30.6     61.2  35.9  4    3.1e-07 ***
## ---
## Signif. codes:  0 '***' 0.001 '**' 0.01 '*' 0.05 '.' 0.1 ' ' 1
\end{verbatim}

\begin{Shaded}
\begin{Highlighting}[]
\FunctionTok{summary}\NormalTok{(lm1)}
\end{Highlighting}
\end{Shaded}

\begin{verbatim}
## Linear mixed model fit by REML ['lmerMod']
## Formula: acidity ~ starter + (1 | batch) + (1 | position)
##    Data: data
## 
## REML criterion at convergence: 61.6
## 
## Scaled residuals: 
##     Min      1Q  Median      3Q     Max 
## -2.1067 -0.4404 -0.0575  0.5405  1.5203 
## 
## Random effects:
##  Groups   Name        Variance Std.Dev.
##  batch    (Intercept) 0.82964  0.9108  
##  position (Intercept) 0.00819  0.0905  
##  Residual             0.54602  0.7389  
## Number of obs: 25, groups:  batch, 5; position, 5
## 
## Fixed effects:
##             Estimate Std. Error t value
## (Intercept)    8.424      0.526   16.01
## starter2      -0.150      0.467   -0.32
## starter3      -0.980      0.467   -2.10
## starter4       2.810      0.467    6.01
## starter5      -0.484      0.467   -1.04
## 
## Correlation of Fixed Effects:
##          (Intr) strtr2 strtr3 strtr4
## starter2 -0.444                     
## starter3 -0.444  0.500              
## starter4 -0.444  0.500  0.500       
## starter5 -0.444  0.500  0.500  0.500
\end{verbatim}

\begin{Shaded}
\begin{Highlighting}[]
\FunctionTok{par}\NormalTok{(}\AttributeTok{mfrow=}\FunctionTok{c}\NormalTok{(}\DecValTok{1}\NormalTok{,}\DecValTok{1}\NormalTok{))}
\FunctionTok{plot}\NormalTok{(}\FunctionTok{fitted}\NormalTok{(lm1), }\FunctionTok{residuals}\NormalTok{(lm1))}
\end{Highlighting}
\end{Shaded}

\includegraphics{ass1_files/figure-latex/unnamed-chunk-26-1.pdf}

\textcolor{red}{Comment?} No structure in data can be seen?

\end{document}
